%This is my super simple Real Analysis Homework template

\documentclass{article}
\usepackage[utf8]{inputenc}
\usepackage[english]{babel}
\usepackage{amsmath,bm}
\usepackage{minted}
\usepackage[]{tcolorbox}
\usepackage[]{amsthm} %lets us use \begin{proof}
\usepackage[]{amssymb} %gives us the character \varnothing
\usepackage{mathtools}
\DeclareMathOperator{\sech}{sech}
\DeclarePairedDelimiter\ceil{\lceil}{\rceil}
\DeclarePairedDelimiter\floor{\lfloor}{\rfloor}

\title{Homework 2}
\author{Guilherme Albertini}
\date\today
%This information doesn't actually show up on your document unless you use the maketitle command below

\begin{document}
\maketitle %This command prints the title based on information entered above

%Section and subsection automatically number unless you put the asterisk next to them.
\section*{Theory}
\subsection*{Problem 1.1: Convolutional Neural Netoworks}
\begin{enumerate}
  \item Given an input image of dimension $21 \times 12$, what will be output
        dimension after applying a convolution with $4 \times 5$ kernel, stride
        of 4,
        and no padding?
        \begin{tcolorbox}
          \begin{flalign*}
            5 \times 2
          \end{flalign*}
        \end{tcolorbox}

  \item Given an input of dimension $C \times H \times W$ what will be the
        dimension of the output of a convolutional layer with kernel of size $K
          \times
          K$, padding P, stride S, dilation D, and F filters. Assume that H
        $\geq$ K, W
        $\geq$ K.
        \begin{tcolorbox}
          Define Padding along height on top	  $P_{H1}$\\

          Define Padding along height on bottom $P_{H2}$\\

          Define Padding along width on left	  $P_{W1}$\\

          Define Padding along width on right	$P _{W2}$\\

          Define Kernel width		 $K_{H}$\\

          Define Kernel height		  $K_{W}$\\

          Define Stride horizontal		  $S_{W}$\\

          Define Stride vertical		  $S_{H}$\\

          Define Batch Count			  $B$\\

          \begin{flalign*}
                   & \text{Effect of adding padding and applying kernel to
              dimensions:}
            \\
            H_P    & = P_{H1} + P_{H2} +
            H                                                              \\
            W_P    & = P_{W1} + P_{W2} +
            W                                                              \\
            H_{PK} & = H_1 - [D_H(K_H - 1)+1]                              \\
                   & =P_{H1} + P_{H2} + H - [D_H(K_H - 1)+1]
            \\
            W_{PK} & = W_1 - [D_W(K_W - 1)+1]                              \\
                   & =P_{W1} + P_{W2} + W - [D_W(K_W - 1)+1]
            \\
          \end{flalign*}
        \end{tcolorbox}
        \begin{tcolorbox}
          \begin{flalign*}
                    & \text{Considering stride to
            dimensions:}                                                    \\
            H_{PKS} & = \floor*{\frac{H_P -
            [D_H(K_H - 1)+1]+S_H}{S_H}}
            \\
                    & =\floor*{\frac{P_{H1} + P_{H2} + H - [D_H(K_H - 1)+1]}{S_H}}+1 \\
            W_{PKS} & = \floor*{\frac{W_P -
            [D_W(K_W - 1)+1]+S_W}{S_W}}                                       \\
                    & =\floor*{\frac{P_{W1} + P_{W2} + W - [D_W(K_W - 1)+1]}{S_W}}+1
          \end{flalign*}
        \end{tcolorbox}
        \begin{tcolorbox}
          We can make simplifcations that I think are implied here:
          \begin{flalign*}
            S & = S_W = S_H                         \\
            D & = D_W = D_H                         \\
            K & = K_W = K_H                         \\
            B & = 1                                 \\
            P & = P_{W1} + P_{W2} = P_{H1} + P_{H2} \\
          \end{flalign*}
          \begin{flalign*}
            W & = \frac{2P + W - [D(K - 1)+1]}{S}+1                       \\
            H & = \frac{2P + H - [D(K - 1)+1]}{S}+1                       \\
              & \text{Thus the output dimension is: }                     \\
               F \times &\left( \floor*{\frac{2P + H - [D(K - 1)+1]}{S}}+1 \right)
            \times \left(\floor*{\frac{2P + W - [D(K - 1)+1]}{S}}+1	 \right)
          \end{flalign*}
        \end{tcolorbox}
  \item Let's consider an input $x[n] \in \mathbb{R}^5$, with $1 \leq n \leq 7$, e.g. it is a length 7 se-
  quence with 5 channels. We consider the convolutional layer $f_W$ with one filter, with kernel size 3, stride of 2, no dilation, and no padding. The only
  parameters of the convolutional layer is the weight$ W, W \in \mathbb{R}^{1 \times 5 \times 3}$ and there is no bias and no non-linearity.
        \begin{enumerate}
          \item What is the dimension of the output $f_W(x)$? Provide an expression for the value of elements of the convolutional layer output $f_W(x)$.
          Example answer format here and in the following sub-problems: $f_W(x) \in \mathbb{R}^{42 \times 42 \times 42}, f_W(x)[i, j,k] = 42.$
        \end{enumerate}

  \item Show

\end{enumerate}

%If you want centered math on its own line, you can use a slash and square
%bracket.\\
%\[
%  \left \{
%  \sum\limits_{k=1}^\infty l(I_k):A\subseteq \bigcup_{k=1}^\infty \{I_k\}
%  \right \}
%\]
%The left and right commands make the brackets get as big as we need them to
%be.

\clearpage %Gives us a page break before the next section. Optional.
\subsection*{Problem 1.2: Recurrent Neural Networks}

$\sigma(z)=\frac{1}{1+\exp(-z)}$.
\begin{enumerate}
  \item If you want

  \item Now
\end{enumerate}

\subsection*{Problem 1.3: Debugging Loss Curves}
%
\begin{enumerate}
  \item Why is softmax actually softargmax?
\end{enumerate}

\end{document}