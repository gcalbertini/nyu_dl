\documentclass{article}
% Setting
\usepackage{fouriernc}
\usepackage[T1]{fontenc}

% Formatting
\setlength{\parskip}{5pt}
\setlength{\parindent}{0pt}

% Text
\usepackage{tcolorbox}

% Math
\usepackage{mathtools, amssymb, bm, amsthm}
\usepackage{bbm}
% original mathcal
\DeclareMathAlphabet{\mathcal}{OMS}{zplm}{m}{n}

% Math Theorem

% Math Shortcut
\DeclareMathOperator{\sign}{sgn}
\DeclareMathOperator{\samax}{softargmax}
\DeclareMathOperator{\smax}{softmax}
\DeclareMathOperator{\smin}{softmin}
\DeclareMathOperator{\diag}{diag}
\DeclareMathOperator*{\argmax}{\arg\max}
\DeclareMathOperator*{\argmin}{\arg\min}
\newcommand{\matr}[1]{{#1}}     % ISO complying version
\newcommand{\vect}[1]{{#1}}     % ISO complying version

% Number sets
\newcommand{\N}{\mathbb{N}} % Natural Numbers
\newcommand{\Z}{\mathbb{Z}} % Integers
\newcommand{\Q}{\mathbb{Q}} % Quotient
\newcommand{\R}{\mathbb{R}}	% Real Numbers
\newcommand{\E}{\mathbb{E}}	% Real Numbers
% \newcommand{\C}{\mathbb{C}} % Complex Numbers

% ML
\newcommand{\cL}{\mathcal{L}}
\newcommand{\cN}{\mathcal{N}}

\newcommand{\vx}{{x}}
\newcommand{\vy}{{y}}
\newcommand{\vb}{{b}}
\newcommand{\vz}{{z}}
\newcommand{\mW}{{W}}

\newcommand{\pdv}[2]{\frac{\partial #1}{\partial #2}}


\usepackage{hyperref}
\hypersetup{
    colorlinks=true,
    linkcolor=blue,
    filecolor=magenta,      
    urlcolor=cyan,
}


% Scientific notation

%Drawing


%Algorithm
\usepackage[ruled,vlined]{algorithm2e}

% Other
\usepackage[shortlabels]{enumitem}
\usepackage{cleveref}

\usepackage{xcolor}

\title{Homework 3: Energy-Based Models}
\author{CSCI-GA 2572 Deep Learning}
\date{Fall 2024}

\begin{document}

\maketitle

The goal of homework 3 is to test your understanding of Energy-Based Models, and to show you one application in structured prediction.

In the theoretical part, we'll mostly test your intuition. You'll need to write brief answers to questions about how EBMs work. In part 2, we will implement a simple optical character recognition system. 

In part 1, you should submit all your answers in a pdf file. As before, we recommend using \LaTeX. 

For part 2, you will implement some neural networks by adding your code to the provided ipynb file.

The due date of homework 3 is 11:55pm \texttt{10/20}.
Submit the following files in a zip file \texttt{your\_net\_id.zip} through NYU classes:
\begin{itemize}
\item \texttt{hw3\_theory.pdf}
\item \texttt{hw3\_impl.ipynb}
\end{itemize}

The following behaviors will result in penalty of your final score:
\begin{enumerate}
\item 10\% penalty for submitting your file without using the correct naming format (including naming the zip file, PDF file or python file wrong, adding extra files in the zip folder, like the testing scripts in your zip file). 
\item 10\% penalty for every extra day of lateness. Up to 4 days max (after that we won't accept submission).
\item 20\% penalty for code submission that cannot be executed following the steps we mentioned.
\end{enumerate}

\section{Theory (50pt)}

\subsection{Energy Based Models Intuition (15pts) }
This question tests your intuitive understanding of Energy-based models and their properties. 
\begin{enumerate}[(a)]

\item (1pts) How do energy-based models allow for modeling situations where the mapping from input $x_i$ to output $y_i$ is not 1 to 1, but 1 to many, or even 1 to an infinite continuum of $y$?
\begin{tcolorbox}
    We are mapping pairs $(x,y)$ to a scalar energy value
	and find that the most likely values of $y$ have a low
	$F(x,y)$. We then observe that, for each $x$, we can have several different
	values $y$ that have this low energy.

    Note a good definition from DeepAI: ``Energy-Based
		      Models (EBMs) discover data dependencies by applying a
		      measure of
		      compatibility (scalar energy) to each configuration of
		      the variables. For a model to make a prediction or
		      decision (inference)
		      it needs to set the value of observed variables to 1
		      and finding values of the remaining variables that
		      minimize that “energy”
		      level.
		      In the same way, machine learning consists of
		      discovering an energy function that assigns low energies
		      to the correct values
		      of the remaining variables,
		      and higher energies to the incorrect values. A
		      so-called “loss functional,” that is minimized during
		      training, is used to
		      measure the quality of the energy functions. Within this
		      framework, there are
		      many energy functions and loss functionals allows
		      available to design different
		      probabilistic and non-probabilistic statistical
		      models.''\\
		      Words from Alf: ``We would like the energy function to
		      be smooth and differentiable so that we can use it to
		      perform the
		      gradient-based method for inference. In order to perform
		      inference, we search
		      this function using gradient descent to find compatible
		      y's. There are many
		      alternate methods to gradient methods to obtain the
		      minimum.''
\end{tcolorbox}

\item (1pts) How do energy-based models differ from models that output probabilities?
\begin{tcolorbox}
    As is the key to their flexibility, we need not concern
		      ourselves
		      with normalization as EBMs output an unnormalized scalar
		      (score) of $F(x,y)$ as
		      opposed to conditional probabilities (i.e
		      $\mathbb{P}(y|x)$ would later require
		      an estimate of normalization).
\end{tcolorbox}

\item  (2pts) How can you use energy function $F_W(x, y)$ to calculate a probability $p(y \mid x)$?
\begin{tcolorbox}
    We can view energies as unnormalised negative log
		      probabilities, and use Gibbs-Boltzmann distribution to
		      convert from energy to
		      probability (with normalization and calibrated $\beta$):
		      \begin{flalign*}
			      \mathbb{P}(y|x)=\frac{\exp(-\beta
				      F(x,y))}{\int_{y'}\exp(-\beta F(x,y'))}
		      \end{flalign*}
		      Note: $\beta$ is postive constant and larger values
		      produce models with more variance whereas smoother ones
		      are produced with
		      smaller values.
\end{tcolorbox}


\item (1pt) Is there a way to control the smoothness of the probability distribution $p(y|x)$ estimated from the energy function $F_W(x,y)$? How do we reduce the variance of $p(y|x)$?
\begin{tcolorbox}
    The temperature affects the smoothness of the distribution: higher temperatures make the distribution smoother by spreading the probabilities more evenly across the possible outcomes $y$.
    Lower temperatures make the distribution sharper by increasing the contrast between higher and lower probability outcomes. Accordingly, variance can be reduced by increasing the temperature, which makes the distribution smoother and reduces overfitting to specific outcomes.
\end{tcolorbox}

\item (2pts) What are the roles of the loss function and energy function? 

\begin{tcolorbox}
    The energy function is a measure of incompatibility
		      between variables (for us, usually the input $x$ and
		      output $y$) whereas the
		      loss function is used to mold the energy function (we
		      minimize loss to end up
		      with a well-behaved energy function). Note that the cost
		      is how far prediction
		      $\hat{y}$ is from target $y$. As Yann mentions: A loss
		      functional, minimized
		      during learning, is used to measure the quality of the
		      available energy
		      functions. A distinction should be made between the
		      energy function, which is
		      minimized by the inference process, and the loss
		      functional, which is minimized
		      by the learning process.
\end{tcolorbox}

\item (2pts) What problems can be caused by using only positive examples for energy (pushing down energy of correct inputs only)? How can it be avoided?
\begin{tcolorbox}
    We may get a case of having energy be 0 everywhere,
    which is a valid minimization of the (flat) energy
    surface under this constraint. As this flat model can
    reach every location of
    the space, the distance between any two points (such as
    the length of the
    latent vector spanning the embedded model manifold's
    reconstructed
    $\tilde{y}=Wz$ to observed target $y$) is 0, hence at
    the minimum energy by default. To avoid this state of
    collapse
    case, we can augment $y
        = [1, y]^{T}$ to give an additional degree of
    freedom to dictionary $W=[1, W]^{T}$, so
    that we can now intersect any point in the 2D space but
    only at those points
    located at the specific height (here, 1) that gives the
    minimum energy near 0.
\end{tcolorbox}


\item 
(2pts) Briefly explain the three methods that can be used to shape the energy function.
\begin{tcolorbox}
    Regularization Methods: if the latent variable z is too
		      expressive power in producing the final prediction
		      $\tilde{y}$ then every true
		      output $y$ will be a perfect reconstruction from input
		      $x$ at the optimized
		      latent $\check{z}$. We can then limit the volume of space
		      of $z$ (say, with L1 loss to promote sparsity) and therby
		      reduce the regions of
		      $y$ with low energy, preventing the case of getting
		      energy 0 everywhere.\\\\
		      Contrastive Methods: Push down the energy of training
		      data points, $F(X_i,Y_i)$, while pushing up energy on
		      everywhere else,
		      $F(X_i,Y')$.\\ \\
		      Architectural Methods: The manifold is of lower
		      dimension than the ambient space so the data cannot be
		      reconstructred
		      perfectly. Autoencoders can reduce the dimensionality of
		      the input in the
		      hidden layer (under-complete) and thus cannot reconstruct
		      data perfectly,
		      preventing collapse. If the hidden space is
		      over-complete, we can use encoding
		      with higher dimensionality than the input to make
		      optimization easier.
\end{tcolorbox}


\item (2pts) Provide an example of a loss function that uses negative examples. The format should be as follows $\ell_\text{example}(x, y, W) = F_W(x, y)$.

\begin{tcolorbox}
    \begin{flalign*}
        \ell_{NLL}(x,y,W) & = -\log \left(
        \frac{\exp(-\beta F(x,y))}{\int_{y'} \exp(-\beta
        F(x,y'))}\right)                   \\
    \end{flalign*}
\end{tcolorbox}

\item (2pts) Say we have an energy function $F(x, y)$ with images $x$, classification for this image $y$.
	Write down the mathematical expression for doing inference given an input $x$.
	Now say we have a latent variable $z$, and our energy is $G(x, y, z)$.
	What is the expression for doing inference then?

    \begin{tcolorbox}
        \begin{flalign*}
            \check{y}            & = \arg \min_{y} {F(x,y)}
            \\
            \check{z}, \check{y} & = \arg \min_{y,z}
            {G(x,y,z)}
        \end{flalign*}
    \end{tcolorbox}



\end{enumerate}


\subsection{Negative log-likelihood loss (20 pts) }

Let's consider an energy-based model we are training to do classification of input between n classes. $F_W(x, y)$ is the energy of input $x$ and class $y$. We consider n classes: $y \in \{1, \dots, n\}$.

\begin{enumerate}[(i)]
\item (2pts) For a given input $x$, write down an expression for a Gibbs distribution over labels $p(y|x)$ that this energy-based model specifies. Use $\beta$ for the constant multiplier.
\begin{tcolorbox}
    \begin{flalign*}
        F_{\beta}(x,y)=-\frac{1}{\beta} \log \int_{z}
        \exp(-\beta G(x,y,z))
    \end{flalign*}
\end{tcolorbox}


\item (5pts) Let's say for a particular data sample $x$, we have the label $y$. Give the expression for the negative log likelihood loss, i.e. negative log likelihood of the correct label (show step-by-step derivation of the loss function from the expression of the previous subproblem). For easier calculations in the following subproblem, multiply the loss by $\frac{1}{\beta}$.
\begin{tcolorbox}
    \begin{flalign*}
        \ell_{NLL}(x,y,W) & = -\log \left(
        \frac{\exp(-\beta
                F_W(x,y))}{\int_{y'} \exp(-\beta
                F_W(x,y'))}\right)
        \\
                          & =\log \left(\int_{y'}
        \exp(-\beta F_W(x,y'))\right) -\log(\exp(-\beta F_W(x,y))
        \\
        & = \log \left(\int_{y'} \exp(-\beta F_W(x,y'))\right) +\beta F_W(x,y))\\
                          & \implies \frac{1}{\beta}\log
        \left(\int_{y'}
        \exp(-\beta F_W(x,y'))\right) +  F_W(x,y)
    \end{flalign*}
    Where the last step was from dividing by $\beta$.
\end{tcolorbox}

\item (8pts) Now, derive the gradient of that expression with respect to $W$ (just providing the final expression is not enough). Your final answer may contain the expression $\frac{\partial F_W(...)}{\partial W}$. Why can it be intractable to compute it, and how can we get around the intractability? 
\begin{tcolorbox}
    \begin{flalign*}
        \frac{\partial \ell_NLL(x,y,W)}{\partial W} & =
        \frac{\partial F_W(x,y)}{\partial W} +
        \frac{1}{\beta} \frac{\partial \log
            \int_{y'} \exp(-\beta
            F_W(x,y'))}{\partial W}
        \\
                                                    & =
        \frac{\partial F_W(x,y)}{\partial W} +
        \frac{1}{\beta}\frac{\frac{\partial \int_{y'}
                \exp(-\beta F_W(x,y'))}{\partial
                W}}{\int_{y'}\exp(-\beta
            F_W(x,y'))}
        \\
        = \frac{\partial F_W(x,y)}{\partial W}      & +
        \frac{1}{\beta}\frac{1}{\int_{y'}\exp(-\beta
            F_W(x,y'))}\int_{y'} -\beta
        \exp(-\beta F_W(x,y')) \frac{\partial
            F_W(x,y')}{\partial W}
        \\
        =\frac{\partial F_W(x,y)}{\partial W}
                                                    &
        -\int_{y'} \frac{\exp(-\beta
            F_W(x,y')}{\int_{y''} \exp(-\beta F(x,
            y''))}\frac{\partial F(x,y')}{\partial W}
        \\
                                                    & =
        \frac{\partial F_W(x,y)}{\partial W} -
        \int_{y'} \mathbb{P}_{\beta}(y'|x)\frac{\partial
            F_W(x,y')}{\partial W}
    \end{flalign*}
    It may be intractable to compute the integral term over
    all $y' \in Y$; we can get around this by perhaps by
    applying Markov chain
    Monte Carlo (MCMC) methods, sequential Monte Carlo (SMC)
    methods, importance
    sampling, or the forward-backward algorithm.
\end{tcolorbox}


\item (5pts) Explain why negative log-likelihood loss pushes the energy of the correct example to negative infinity, and all others to positive infinity, no matter how close the two examples are, resulting in an energy surface with really sharp edges in case of continuous $y$ (this is usually not an issue for discrete $y$ because there's no distance measure between different classes).
\begin{tcolorbox}
    NLL pulls all negative examples up with force
    proportional
    to the probability of the $y'$ in question (see Energy
    Classification slides).
    This is not a distance metric so proximity of two
    examples is irrelevant. For
    the continuous case, this force remains the same
    regardless of proximity of
    examples so we end up with sharp edges over the energy
    surface.
\end{tcolorbox}

\end{enumerate}

\subsection{Comparing Contrastive Loss Functions (15pts)}

In this problem, we're going to compare a few contrastive loss functions. We are going to look at the behavior of the gradients, and understand what uses each loss function has. In the following subproblems, $m$ is a margin, $m \in \R$, $x$ is input, $y$ is the correct label, $\bar y$ is the incorrect label. Define the loss in the following format: $\ell_{example}(x, y, \bar y, W) = F_W(x, y)$.

\begin{enumerate}[(a)]
\item (2pts) \textbf{Simple loss function} is defined as follows:

$$
\ell_\text{simple}(x, y, \bar y, W) = \left[ F_W(x, y)\right]^+ + \left[m - F_W(x, \bar y)\right]^+
$$

Where $[z]^+ = max(0, z)$ 

Assuming we know the derivative $\pdv{F_W(x, y)}{W}$ for any $x, y$, give an expression for the partial derivative of the $\ell_\text{simple}$ with respect to $W$.

\begin{tcolorbox}
    \begin{flalign*}
        \frac{\partial [[F_W(x,y)]^{+}] + [m -
        F_{W}(x,\bar{y})]]^{+}}{\partial W}          & =
        \frac{\partial [F_W(x,y)]^{+}]}{\partial
            W} + \frac{[m - F_{W}(x,\bar{y})]^{+}}{\partial
        W}                                              \\
        \frac{\partial [F_W(x,y)]^{+}]}{\partial W} & =
        \begin{cases}
            \frac{\partial F_W(x,y)}{\partial W} &
            \text{if
            } F_W(x,y) > 0
            \\
            0                                    &
            \text{otherwise }
        \end{cases}
        \\
        \frac{\partial [m -
                F_{W}(x,\bar{y})]^{+}}{\partial W}
                                                    & =
        \begin{cases}
            \frac{-\partial
            F_W(x,\bar{y})}{\partial W} & \text{if }
            F_W(x,\bar{y}) < m                       \\
            0                           &
            \text{otherwise }
        \end{cases}
    \end{flalign*}\\
\end{tcolorbox}

\item (2pts) \textbf{Hinge Loss} is defined as follows:
$$
\ell_\text{hinge}(x, y, \bar y, W) = \left[m + F_W(x,y) - F_W(x, \bar y)\right]^+
$$

Assuming we know the derivative $\pdv{F_W(x, y)}{W}$ for any $x, y$, give an expression for the partial derivative of the $\ell_\text{hinge}$ with respect to $W$.
\begin{tcolorbox}

\[
\pdv{\ell_\text{hinge}(x, y, \bar y, W)}{W} = \pdv{F_W(x, y)}{W} - \pdv{F_W(x, \bar y)}{W}.
\]

\[
\pdv{\ell_\text{hinge}(x, y, \bar y, W)}{W} = 
\begin{cases}
\pdv{F_W(x, y)}{W} - \pdv{F_W(x, \bar y)}{W}, & \text{if } m + F_W(x, y) - F_W(x, \bar y) > 0, \\
0, & \text{otherwise}.
\end{cases}
\]
\end{tcolorbox}


\item (2pts) \textbf{Log loss} is defined as follows:

$$
\ell_\text{log}(x, y, \bar y, W) = \log \left(1 +  e^{F_W(x, y) - F_W(x, \bar y)} \right)
$$

Assuming we know the derivative $\pdv{F_W(x, y)}{W}$ for any $x, y$, give an expression for the partial derivative of the $\ell_\text{log}$ with respect to $W$.
\begin{tcolorbox}
    \begin{flalign*}
        \frac{\partial
            \log(1+\exp(F_W(x,y)-F_W(x,\bar{y})))}{\partial
        W} = \\
        \frac{\left(
            \exp(F_W(x,y)-F_W(x,\bar{y}))\right)\left(\frac{\partial F_W(x,y)}{\partial W}
            - \frac{\partial F_W(x,\bar{y})}{\partial
                W}\right)}{1+\exp(F_W(x,y)-F_W(x,\bar{y}))}
        \\
        =\frac{\exp(F_W(x,y))\left(\frac{\partial
                F_W(x,y)}{\partial W} -
            \frac{\partial F_W(x,\bar{y})}{\partial
                W}\right)}{\exp(F_W(x,y))+\exp(F_W(x,\bar{y}))}
    \end{flalign*}
\end{tcolorbox}


\item (2pts) \textbf{Square-Square loss} is defined as follows:

$$
\ell_\text{square-square}(x, y, \bar y, W) = \left(\left[ F_W(x, y)\right]^+ \right)^2 + \left( \left[m - F_W(x, \bar y)\right]^+ \right)^2
$$

Assuming we know the derivative $\pdv{F_W(x, y)}{W}$ for any $x, y$, give an expression for the partial derivative of the $\ell_\text{square-square}$ with respect to $W$.
\begin{tcolorbox}
    \begin{flalign*}
        \frac{\partial [(F_W(x,y))^{+})^2 +
        ((m-F_W(x,\bar{y}))^{+})^2]}{\partial W} = \\
        \frac{\partial (F_W(x,y)^{+})^2}{\partial W} +
        \frac{\partial
            ((m-F_W(x,\bar{y}))^{+})^2}{\partial W}
    \end{flalign*}
    \begin{flalign*}
        \frac{\partial (F_W(x,y))^{+})^2}{\partial W} & =
        \begin{cases}
            2F_W(x,y)\frac{\partial
            F_W(x,y)}{\partial W} & \text{if }
            F_W(x,y) > 0                              \\
            0                     & \text{otherwise }
        \end{cases}                \\
        \frac{\partial
            ((m-F_W(x,\bar{y}))^{+})^2}{\partial W}
                                                      & =
        \begin{cases}
            -2(m-F_W(x,\bar{y}))\frac{\partial
            F_W(x,\bar{y})}{\partial W} & \text{if }
            F_W(x,\bar{y}) < m                       \\
            0                           &
            \text{otherwise }
        \end{cases}
    \end{flalign*}
\end{tcolorbox}

\item (7pts) \textbf{Comparison.} 
\begin{enumerate}[(i)]
    \item (2pts) Explain how NLL loss is different from the three losses above.
    \begin{tcolorbox}
        NLL loss can be applied in the continuous case, in
        which it can prove an intractable calculation. Also note that NLL loss pulls up
        on all negative examples at a time.
    \end{tcolorbox}
    \item (2pts) The hinge loss $\left[ F_W(x, y) - F_W(x, \bar y) + m \right]^+$ has a margin parameter $m$, which gives 0 loss when the positive and negative examples have energy that are $m$ apart.
    		 The log loss is sometimes called a "soft-hinge" loss. Why? What is the advantage of using a soft hinge loss?
             \begin{tcolorbox}
                By taking a margin of m, we notice the following:
                \begin{flalign*}
                   \exp(F_W(x,y)-F_W(x,\bar{y})+m) \propto  (1+\exp(F_W(x,y)-F_W(x,\bar{y})))\\
                  \implies m \propto \log (1+\exp(F_W(x,y)-F_W(x,\bar{y}))) - (F_W(x,y)-F_W(x,\bar{y}))\\
                  \therefore m+F_W(x,y)-F_W(x,\bar{y}) \propto  \log (1+\exp(F_W(x,y)-F_W(x,\bar{y})))
                \end{flalign*}
  
                 Thus we `soften' the exponential representation by using log to arrive at log loss, noting that it
                 is proportional to hinge loss for this given margin. The soft-hinge loss stabilizes the exponential (hinge loss) 
                 representation and is resilient to over/underflow. We also guarantee that the 
                 output won't be at a vastly different scale than the input using the log.
            \end{tcolorbox}
    \item (2pts) How are the simple loss and square-square loss different from the hinge/log loss?
    \begin{tcolorbox}
        Square-square loss pushes the positive examples' energy
        towards 0 and pulls negative examples' energy away from 0 quadratically,
        whereas simple loss does this linearly. In comparison, hinge/log loss concerns
        itself with the difference in energy between the positive and negative
        examples. Hinge loss' positive examples may have lower energy than those of negative by at
        least the margin $m$ so it may not produce energy of 0. Use
        simple loss if the model does not need to be sensitive to outliers we come
        across and use square-square loss if it does.
    \end{tcolorbox}
    \item (1pt) In what situations would you use the simple loss, and in what situations would you use the square-square loss?
\end{enumerate}


\end{enumerate}


\section{Implementation (50pt + 10pt extra credit)}

Please add your solutions to this notebook
\href{https://drive.google.com/file/d/1zDjkA0R8puzTkJsJqGLjqYhLfYLJtAMT/view?usp=sharing}{\texttt{hw3\_impl.ipynb}}
.
\textbf{Plase use your NYU account to access the notebook.} The notebook contains parts marked as \texttt{TODO}, where you should put your code or explanations. The notebook is a Google Colab notebook, you should copy it to your drive, add your solutions, and then download and submit it to NYU Classes. You're also free to run it on any other machine, as long as the version you send us can be run on Google Colab.


\end{document}
